\documentclass[a4paper,twocolumn]{article}
\usepackage{graphicx}
\usepackage{listings}
\usepackage[justification=centering]{caption}
\usepackage{subfigure}  
\usepackage{multirow}
\usepackage{balance}
\lstset{language=Matlab}
\lstset{breaklines}
\lstset{extendedchars=false}

\usepackage{amsmath,amsfonts,amsthm,amssymb}
\theoremstyle{definition}
\newtheorem{thm}{Theorem}
\newtheorem{exmp}{Example}
\newtheorem{defn}{Definition}
\newtheorem{lema}{Lemma}
\newtheorem{prop}{Proposition}
\newtheorem{coro}{Corollary}


\renewcommand{\baselinestretch}{1.25}

%------------setlength----------------%
\setlength{\textwidth}{162mm}
%\setlength{\textheight}{190mm}
\setlength{\textheight}{231mm}
\setlength{\headheight}{-0.1cm}
\setlength{\topmargin}{-0.1cm}
\setlength{\oddsidemargin}{-0cm}
\setlength{\evensidemargin}{-0cm}

\setlength\columnsep{12pt}
\setlength\columnseprule{0.4pt}

\setlength{\parskip}{1mm}
\setlength{\unitlength}{1mm}
\setlength{\parindent}{2em}

\title{Project2 - Mathematical theory}

\author{Li Zhiqi\quad3180103041}

\begin{document}
\maketitle
\section{Problem analysis}
Our goal is to use the multigrid solver to solve the one-dimensional Possion equation
$$
-u^{\prime \prime}(x)=f(x)
$$
on $\Omega = [0,1]$ with Dirichlet boundary condition. The solver are expected to work when facing with different boundary conditions, different restriction and interpolation operators, different stopping criteria and different initial guess. \\
Firstly, we turn the problem into a linear system of equations. For the function
$$
u(x) = \exp(\sin(x)),
$$
we get 
$$
f(x) = \sin(x)\exp(\sin(x))-\cos^2(x)\exp(\sin(x)),
$$
and
$$
u(0) = \alpha = 1,u(1) = \beta = \exp(\sin(1)) .
$$
The situation is similar when boundary conditions are homogeneous.\\
By finite difference method with a uniform gird size $h = \frac{1}{n}$, our discretization of the equation yields a linear system
$$
A\textbf{u} = \textbf{f}
$$
where
$$
A=\frac{1}{h^{2}}\left[\begin{array}{ccccccc}
2 & -1 & & & & & \\
1 & -2 & 1 & & & & \\
& 1 & -2 & 1 & & & \\
& & \ddots & \ddots & \ddots & & \\
& & & 1 & -2 & 1 & \\
& & & & 1 & -2 & 1 \\
& & & & & -1 & 2
\end{array}\right],
$$
\newpage

$$
\mathbf{u}=\left[\begin{array}{c}
U_{1} \\
U_{2} \\
U_{3} \\
\vdots \\
U_{n-2} \\
U_{n-1}
\end{array}\right], \quad \mathbf{f}=\left[\begin{array}{c}
f\left(x_{1}\right)+\frac{\alpha}{h^{2}} \\
f\left(x_{2}\right) \\
f\left(x_{3}\right) \\
\vdots \\
f\left(x_{n-2}\right) \\
f\left(x_{n-1}\right)+\frac{\beta}{h^{2}}
\end{array}\right],
$$
and $x_j = jh , j = 1,2,\cdots,n-1$. We will compute $n-1$ values $U_{1},U_{2},\cdots,U_{n-1}$ where each $U_{j}$ approximates $u(x_j)$.\\
The eigenvalues and eigenvectors of A are
$$
\begin{aligned}
\lambda_{k}(A)&=\frac{4}{h^{2}} \sin ^{2} \frac{k \pi}{2 n} \ , \\
\mathbf{w}_{k, j}&=\sin \frac{j k \pi}{n} ,
\end{aligned}
$$
where $j,k = 1,2,\cdots ,n-1$.
\section{Preparation for multigrid solver}
Before introducing multigrid solver, we need to make some preparations. The following is necessary when using multigrid.
\subsection{Weighted Jacobi method}
Weighted Jacobi method is one of classical iterative methods for solving a linear system $A\textbf{u} = \textbf{f}$, which has the form
$$
\begin{aligned}
\mathbf{u}^{*} &=T \mathbf{u}^{(\ell)}+\mathbf{c} \\
\mathbf{u}^{(\ell+1)} &=(1-\omega) \mathbf{u}^{(\ell)}+\omega \mathbf{u}^{*},
\end{aligned}
$$
where $A = D + L + U$ with $D, L, U$ being the diagonal, lower triangular, and upper tri-angular part of $A$, and $T=-D^{-1}(L+U), \quad \mathbf{c}=D^{-1} \mathbf{f}$.
\newpage
\noindent Dealing with the specific A and $\mathbf{f}$ in the section 1, we have
$$
\begin{aligned}
T_{\omega}&= (1-\omega)I+\omega T \\
&=(1-\omega)I - \omega D^{-1}(A-D) \\
&=I - \omega D^{-1}A \\
&=I - \frac{\omega h^2}{2}A \ ,
\end{aligned}
$$
with the corresponding eigenvalues as
$$
\lambda_k(T_{\omega}) = 1 - 2\omega \sin^2\frac{k\pi}{2n} \ .
$$
In our multigrid calculation, we always set $\omega = \frac{2}{3}$,
which performs much better for HF modes than regular Jacobi method.\\
In summary, we will use the iteration
$$
\mathbf{u}^{(\ell+1)} =T_{\omega} \mathbf{u}^{(\ell)}+\mathbf{c}^{\prime}(\mathbf{f})
$$
as relaxed operator in multigrid cycles, where 
$$
T_{\omega} = I - \frac{h^2}{3}A , \mathbf{c}^{\prime}(\mathbf{f}) = \frac{h^2}{3}\mathbf{f} \ .
$$
\subsection{Restriction and interpolation operators}
Restriction operator $I_h^{2h}$ maps a vector on the fine grid $\Omega^h$ to its counterpart on the coarse grid $\Omega^{2h}$ :
$$
I_{h}^{2 h} v^{h}=v^{2 h} \ .
$$
Two different restriction operator will be used in our calculation:\\
\textbf{Full-weighting} operator is given by
$$
v_{j}^{2 h}=\frac{1}{4}\left(v_{2 j-1}^{h}+2 v_{2 j}^{h}+v_{2 j+1}^{h}\right) \ ,
$$
where $j = 1,2,\cdots,\frac{n}{2} - 1.$\\
\textbf{Injection} operator is given by
$$
v_{j}^{2 h}=v_{2 j}^{h} \ ,
$$
where $j = 1,2,\cdots,\frac{n}{2} - 1.$\\
Correspondingly, interpolation operator $I_{2h}^{h}$ maps a vector on the coarse grid $\Omega^{2h}$ to its counterpart on the fine grid $\Omega^h$   :
$$
I_{2 h}^{h} v^{2h}=v^{h} \ .
$$
\newpage
\noindent Two different interpolation operator will be used in our calculation:\\
\textbf{Linear interpolation} operator is given by
$$
\begin{aligned}
v_{2 j}^{h} &=v_{j}^{2 h} \ , \\
v_{2 j+1}^{h} &=\frac{1}{2}\left(v_{j}^{2 h}+v_{j+1}^{2 h}\right) \ ,
\end{aligned}
$$
where $j = 1,2,\cdots,\frac{n}{2} - 1$, and we take
$$
v_{1}^{h} = \frac{1}{2}v_{1}^{2h},\quad v_{n-1}^{h} = \frac{1}{2}v_{\frac{n}{2} - 1}^{2h}
$$
as values on the boundary.\\
\textbf{Quadratic interpolation} operator is given by
$$
\begin{aligned}
v_{2 j}^{h} &=v_{j}^{2 h} \ , \\
v_{2 j+1}^{h} =\frac{1}{16}(3v_{j-1}^{2h} & +5v_{j}^{2 h}  +5v_{j+1}^{2 h}+3v_{j+1}^{2h} ) \ ,
\end{aligned}
$$
where $j = 1,2,\cdots,\frac{n}{2} - 1$, and we take
$$
\begin{aligned}
v_{1}^{h} &= \frac{5}{16}v_{1}^{2h}+\frac{3}{16}v_{2}^{2h}, \\ v_{3}^{h} &= \frac{5}{16}v_{1}^{2h}+\frac{5}{16}v_{2}^{2h}+\frac{3}{16}v_{3}^{2h} \ ,\\
v_{n-3}^{h} &=\frac{5}{16}v_{\frac{n}{2} - 1}^{2h}+\frac{5}{16}v_{\frac{n}{2} - 2}^{2h}+\frac{3}{16}v_{\frac{n}{2} - 3}^{2h} \ ,\\
v_{n-1}^{h} &=\frac{5}{16}v_{\frac{n}{2} - 1}^{2h}+\frac{3}{16}v_{\frac{n}{2} - 2}^{2h} \ ,
\end{aligned}
$$
as values on the boundary.\\
\section{V-cycle}
In section 2 , we have shown relaxed operator, restriction operator and interpolation operator as preparations for multigrid. Next we will show V-cycle and its effectiveness analysis.
\subsection{V-cycle scheme}
The V-cycle scheme
$$
\mathbf{v}^h \gets \mathrm{VC}^h(\mathbf{v}^h,\mathbf{f}^h,\nu_1,\nu_2) 
$$
solving $A\textbf{u} = \textbf{f}$ is given by\\
(1)Relax $\nu_1$ times on $A^h\textbf{u}^h = \textbf{f}^h$ by Weighted Jacobi method with initial guess $\mathbf{v}^h$. Specifically, calculate $$\mathbf{v}^{(\ell+1)} =T_{\omega}^h \mathbf{v}^{(\ell)}+\mathbf{c}^{\prime}(\mathbf{f}^h)$$
with initial guess $\mathbf{v}^{0} = v^h$, and $T_{\omega}^h,\mathbf{c}^{\prime}(\mathbf{f}^h)$ are listed in the end of section 2.1.
After iterations, assign value to $\mathbf{v}^h$ with $\mathbf{v}^{(\nu_1)}$.\\\\
(2)If $\Omega^h$ is the coarsest grid(We take $n \le 6$ as standard), go to step (4).
Otherwise, calculate 
$$
\begin{aligned}
\mathbf{r}^h &\gets \textbf{f}^h - A^h\textbf{v}^h \ , \\ 
\mathbf{f}^{2h} &\gets I_h^{2h} \mathbf{r}^h \ , \\
\mathbf{v}^{2h} &\gets \mathbf{0} \ , \\
\mathbf{v}^{2h} &\gets \mathrm{VC}^{2h}(\mathbf{v}^{2h},\mathbf{f}^{2h},\nu_1,\nu_2) \ ,
\end{aligned}
$$
where restriction operator $I_h^{2h}$ is selected from full-weighting and injection, which are shown in Section 2.2 .\\\\
(3)Interpolate error back and correct the solution:
$$
\mathbf{v}^{h} \gets \mathbf{v}^{h} +I_{2h}^{h} \mathbf{v}^{2h} \ , 
$$
where interpolation operator $I_{2h}^{h}$ is selected from linear interpolation and quadratic interpolation, which are shown in Section 2.2 .\\\\
(4)Relax $\nu_2$ times on $A^h\textbf{u}^h = \textbf{f}^h$ by Weighted Jacobi method with initial guess $\mathbf{v}^h$. The specific calculation process is the same as step(1). At the end, calculate 
$$
\mathbf{r}^h \gets \textbf{f}^h - A^h\textbf{v}^h \ .
$$
If  $\left \| \mathbf{r}^h \right \| < \epsilon$ or the scheme has reached number of maximum iterations, output $\textbf{v}^h$ as the result, where $\epsilon$ is the accuracy given by users. Otherwise, go to step (2) with above $\mathbf{r}^h$.
\subsection{Effectiveness analysis}
To simplify the analysis, we only consider two-grid V-cycle with number of maximum iterations 1. In fact, the above V-cycle method is just combination and repetition of this simple case.\\
Consider the iteration matrix of two-grid operator in $\Omega^h$ and $\Omega^{2h}$ when acting on the error vector, we have
$$
T G=T_{\omega}^{\nu_{2}}\left[I-I_{2 h}^{h}\left(A^{2 h}\right)^{-1} I_{h}^{2 h} A^{h}\right] T_{\omega}^{\nu_{1}} \ ,         
$$
which comes from the subtraction of the action in initial guess $v^h$ and exact solution $u^h$.\\
Our goal is to estimate the effect of two-grid operator on Fourier modes of different frequencies, especially low-frequency modes. \\
With full-weighting and linear interpolation as restriction and interpolation operator, we have
$$
\begin{aligned}
I_{h}^{2 h} \mathbf{w}_{k}^{h} &=c_{k} \mathbf{w}_{k}^{2 h}:=\cos ^{2} \frac{k \pi}{2 n} \mathbf{w}_{k}^{2 h}  \ ,\\
I_{h}^{2 h} \mathbf{w}_{k^{\prime}}^{h} &=-s_{k} \mathbf{w}_{k}^{2 h}:=-\sin ^{2} \frac{k \pi}{2 n} \mathbf{w}_{k}^{2 h} \ , \\
I_{2 h}^{h} \mathbf{w}_{k}^{2 h}&=c_{k} \mathbf{w}_{k}^{h}-s_{k} \mathbf{w}_{k^{\prime}}^{h} \ ,
\end{aligned}
$$
where $k \in [1,\frac{n}{2}), k'= n-k$. The above formulas come from definitions in section 2.2 and trigonometric identities. \\
By these formulas, we get the action of two-grid operator on invariant subspace $W_k^h = $ span\{$\mathbf{w}_k^h,\mathbf{w}_{k'}^h$\}:
$$
\begin{array}{l}
T G \mathbf{w}_{k}=\lambda_{k}^{\nu_{1}+\nu_{2}} s_{k} \mathbf{w}_{k}+\lambda_{k}^{\nu_{1}} \lambda_{k^{\prime}}^{\nu_{2}} s_{k} \mathbf{w}_{k^{\prime}}  \\
T G \mathbf{w}_{k^{\prime}}=\lambda_{k^{\prime}}^{\nu_{1}} \lambda_{k}^{\nu_{2}} c_{k} \mathbf{w}_{k}+\lambda_{k^{\prime}}^{\nu_{1}+\nu_{2}} c_{k} \mathbf{w}_{k^{\prime}} \ ,
\end{array}
$$
where $\lambda_k$ is the eigenvalue of \ $T_\omega$ in section 2.1 .\\
\textit{Proof}. We only proof the first formula. Firstly consider the case $\nu_1 = \nu_2 = 0$. We have
$$
\begin{aligned}
I_{2 h}^{h} \left(A^{2 h}\right)^{-1} I_{h}^{2 h} A^{h} \mathbf{w}_{k}^{h}&=
I_{2 h}^{h} \left(A^{2 h}\right)^{-1} I_{h}^{2 h} \frac{4 s_{k}}{h^{2}} \mathbf{w}_{k}^{h}\\
&=I_{2 h}^{h} \left(A^{2 h}\right)^{-1} \frac{4 c_{k}s_{k}}{h^{2}} \mathbf{w}_{k}^{2h}\\
&=I_{2 h}^{h} \frac{4 c_{k} s_{k}}{h^{2}} \frac{(2 h)^{2}}{4 \sin ^{2} \frac{k \pi}{n}} \mathbf{w}_{k}^{2 h} \\
&=I_{2 h}^{h} \mathbf{w}_{k}^{2 h} \\
&=c_{k} \mathbf{w}_{k}^{h}-s_{k} \mathbf{w}_{k^{\prime}}^{h} \ ,
\end{aligned}
$$
hence
$$
\left[I-I_{2 h}^{h}\left(A^{2 h}\right)^{-1} I_{h}^{2 h} A^{h}\right] \mathbf{w}_{k}^{h}=s_{k} \mathbf{w}_{k}^{h}+s_{k} \mathbf{w}_{k^{\prime}}^{h} \ .
$$
At last, pre-smoothing incurs a scaling of $\lambda_{k}^{\nu_{1}}$ for both $\mathbf{w}_{k}^{h}$ and $\mathbf{w}_{k^{\prime}}^{h}$, and post-smoothing incurs a scaling of $\lambda_{k}^{\nu_{2}}$ for $\mathbf{w}_{k}^{h}$ and a scaling of $\lambda_{k^{\prime}}^{\nu_{2}}$ for $\mathbf{w}_{k^{\prime}}^{h}$. Hence the first formula holds. The second formula is similar.\qed \\
Through the above derivation, we can show the effectiveness of two-grid operator. All four coefficients $(\lambda_{k}^{\nu_{1}+\nu_{2}} s_{k},\lambda_{k}^{\nu_{1}}\lambda_{k^{\prime}}^{\nu_{2}} s_{k},\lambda_{k^{\prime}}^{\nu_{1}} \lambda_{k}^{\nu_{2}} c_{k},\lambda_{k^{\prime}}^{\nu_{1}+\nu_{2}} c_{k})$ are always small for any $k$. So the two-grid operator performs well for both LF modes and HF modes.\\
As mentioned at the beginning of this section, V-cycle is just the combination and repetition of two-grid operator. So far we have completed the effectiveness analysis.
\section{Full multigrid V-cycle}
\balance
The full multigrid V-cycle scheme
$$
\mathbf{v}^h \gets \mathrm{FMG}^h(\mathbf{f}^h,\nu_1,\nu_2) 
$$
solving $A\textbf{u} = \textbf{f}$ is given by\\
(1)If $\Omega^h$ is the coarsest grid(We take $n \le 6$ as standard), set $\mathbf{v}^h \gets \mathbf{0}$ and go to step (3).
Otherwise, calculate 
$$
\begin{aligned}
\mathbf{f}^{2h} &\gets I_h^{2h} \mathbf{f}^h \ , \\
\mathbf{v}^{2h} &\gets \mathrm{FMG}^{2h}(\mathbf{f}^{2h},\nu_1,\nu_2) \ ,
\end{aligned}
$$
where restriction operator $I_h^{2h}$ is selected from full-weighting and injection.\\\\
(2)Correct by
$$
\mathbf{v}^{h} \gets I_{2h}^{h} \mathbf{v}^{2h} \ , 
$$
where interpolation operator $I_{2h}^{h}$ is selected from linear interpolation and quadratic interpolation.\\\\
(3)Perform a V-cycle with the initial guess $\mathbf{v}^{h}$:
$$
\mathbf{v}^h \gets \mathrm{VC}^h(\mathbf{v}^h,\mathbf{f}^h,\nu_1,\nu_2)  \ .
$$
\\
Since full multigrid V-cycle is based on V-cycle, the effectiveness of full multigrid V-cycle can be similarly analyzed.
\end{document}